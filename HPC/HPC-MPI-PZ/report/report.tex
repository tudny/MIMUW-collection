\documentclass[a4paper,12pt]{article}

\title{ MPI task - report }
\author{ HPC | Aleksander Tudruj }
\date{ 2024-06-25 }

%\pagestyle {empty} %wyłącza numerowanie stron
\usepackage[super]{nth}

\usepackage[T1]{fontenc}
\usepackage[utf8]{inputenc}

\usepackage[usenames,dvipsnames]{xcolor}
\definecolor{Nblue}{RGB}{0, 72, 110}
\definecolor{Lblue}{RGB}{236, 250, 255}

\usepackage[headheight=0cm, left=1cm, right=1cm, top=2cm, bottom=2cm]{geometry}
\usepackage{amsfonts,amsmath}
\usepackage{amsmath,amsthm,amssymb}
\usepackage{gensymb}
\usepackage{enumitem}
\usepackage{stmaryrd}
\usepackage{blindtext}
\usepackage{hyperref}

\newcommand{\stirSet}[2]{\left\{\begin{array}{c}
                                    \!\!#1\\ \!\!#2
\end{array}\!\!\right\}}
\newcommand{\stirPerm}[2]{\left[\begin{array}{c}
                                    \!\!#1\\ \!\!#2
\end{array}\!\!\right]}
\newcommand{\stirBin}[2]{\left\left(\begin{array}{c}
                                        \!\!#1\\ \!\!#2
\end{array}\!\!\right\right)}
\newcommand{\set}[1]{\left\{ #1\right\}}
\newcommand{\floor}[1]{\lfloor #1\rfloor}
\newcommand{\ceil}[1]{\lceil #1\rceil}
\newcommand{\suma}[2]{\sum \limits_{#1}^{#2} }
\newcommand{\infsum}[1]{\sum_{#1}^{\infty}}
\newcommand{\mysuma}[2]{\sum_{#1}^{#2}}
\newcommand{\myroot}[2]{\sqrt[\leftroot{0}\uproot{0}#1]{#2}}

\newcommand{\vect}[1]{\overrightarrow{#1}}
\newcommand{\af}[1]{\text{af}{#1}}
\newcommand{\lin}[1]{\text{lin}{#1}}
\newcommand{\R}[0]{\mathbb{R}}
\newcommand{\C}[0]{\mathbb{C}}
\newcommand{\N}[0]{\mathbb{N}}
\newcommand{\Z}[0]{\mathbb{Z}}
\newcommand{\Q}[0]{\mathbb{Q}}
\newcommand{\mat}[1]{\left[\begin{matrix}
                               #1
\end{matrix}\right]}
\newcommand{\operacje}[1]{\begin{matrix}
                              #1
\end{matrix}}
\newcommand{\row}[0]{\text{R}}
\newcommand{\uklad}[1]{\begin{cases}
                           #1
\end{cases}}
\newcommand{\la}[0]{\left\left(}
\newcommand{\ra}[0]{\right\right)}

\newcommand{\myeq}[1]{\mathrel{\stackrel{\makebox[0pt]{\mbox{\normalfont\tiny #1}}}{=}}}

\usepackage { graphicx }

\usepackage{listings}
\usepackage{color}

\usepackage{multicol}

\definecolor{dkgreen}{rgb}{0,0.6,0}
\definecolor{gray}{rgb}{0.5,0.5,0.5}
\definecolor{mauve}{rgb}{0.58,0,0.82}

\lstset{frame=tb,
    language=C++,
    aboveskip=3mm,
    belowskip=3mm,
    showstringspaces=false,
    columns=flexible,
    basicstyle={\small\ttfamily},
    numbers=none,
    numberstyle=\tiny\color{gray},
    keywordstyle=\color{blue},
    commentstyle=\color{dkgreen},
    stringstyle=\color{mauve},
    breaklines=true,
    breakatwhitespace=true,
    tabsize=2
}

\usepackage{caption}
\usepackage{subcaption}


\newcommand{\plottimes}[4]{
    \begin{figure}
        \begin{subfigure}{.5\textwidth}
            \centering
            \includegraphics[width=.9\linewidth]{img/#1}
        \end{subfigure}%
        \begin{subfigure}{.5\textwidth}
            \centering
            \includegraphics[width=.9\linewidth]{img/#2}
        \end{subfigure}
        \begin{subfigure}{.5\textwidth}
            \centering
            \includegraphics[width=.9\linewidth]{img/#3}
        \end{subfigure}%
        \begin{subfigure}{.5\textwidth}
            \centering
            \includegraphics[width=.9\linewidth]{img/#4}
        \end{subfigure}
    \end{figure}
}

\begin{document}
    \maketitle

    \tableofcontents
    \newpage

    \section{ Distribution }

    Both input matrices are given as text files available for the first process.
    Every algorithm uses the same distribution method.
    The method is pretty simple -- for a single the first process reads
    the whole matrix into memory, calculates the ranges for the process grid
    and sends the data row by row (in the process grid).
    It sends them asynchronously to the other processes in a single row,
    as the sub-matrices can take too much space in the memory.

    Ranges are calculated as described in the task description.
    No two ranges lengths can differ by more than 1.
    Adequate unit tests are provided in the code.
    This ensures data is distributed evenly among the processes.

    One note regarding sending a matrix. A matrix object is represented as a tuple
    of six elements:
    \begin{itemize}
        \item uint32\_t rows
        \item uint32\_t cols
        \item uint32\_t nnz
        \item array of double values
        \item array of uint32\_t col indices
        \item array of uint32\_t row offsets
    \end{itemize}

    Those arrays can take more space than MPI allows to send in a single request.
    That is why the data is split into batches and sent in multiple non-blocking requests.
    First request is always a matrix metadata (rows, cols, nnz) and the rest are the data.
    This is important in SUMMA3D implementation.

    \section{ Summa2D }

    Algorithm SUMMA2D is implemented as described in the provided paper.
    In order to keep the implementation simple, the indirect matrix results
    are kept in hash map (std::unordered\_map). This enables to implement
    the local multiplication algorithm, that does approximately the same
    number of steps as the number of hits in the multiplication --
    by hit I mean a pair of elements that are multiplied together.

    The key in the hash map is a uint64\_t number that represents
    the binary concatenation of the row and column indices that are uint32\_t.

    Maybe I am not sure whether this is faster than concatenation of the
    matrices, but it is definitely faster than concatenation on the fly.

    If no verbose flag is set, the result matrix is not loaded into master process.
    If g\_value flag is set, the result is collected using MPI\_Gather with sum operation.

    \section{ Summa3D }

    Algorithm SUMMA3D is implemented as described in the provided paper, as well.
    It uses SUMMA2D algorithm directly by leveraging MPI Comm split on layer as a key.

    Due to the problem described in the Distribution section, the data is sent in batches.
    I decided not to use MPI provided AllToAll function, as it would require to send
    the data in a single request, or do some mambo jumbo with MPI\_Datatypes.

    Instead, I implemented the all to all by hand using the following schema:
    Let's say we are process $(i, j, l)$ in the grid and want to send data
    to all processes in the same layer, and receive data from all processes in the same layer.
    We go in a loop of $o_{ffset} = 1, ..., layers$.
    In each iteration we send data to $(i, j, l - o_{ffset})$ and receive data from $(i, j, l + o_{ffset})$
    -- all the arithmetic is done modulo the grid size. Sends are done asynchronously,
    and receives are done synchronously.

    \section{ Summa3D balanced }

    Unfortunately, I did not have time to implement the balanced version of the SUMMA3D algorithm.
    I only had an idea how to approach this problem painlessly during my morning shower.
    As it is not implemented feel free to ignore this section.

    The idea is based on the following observation:

    Let $A, B \in \R^{n \times n}$ and let $P \in \R^{n \times n}$ be a permutation matrix.
    Let $C = A \cdot B$. As $P$ is a permutation matrix it is orthogonal, so $P^T = P^{-1}$.
    Then $C = A \cdot B = A \cdot I \cdot B = A \cdot P^T \cdot P \cdot B = (A \cdot P^T) \cdot (P \cdot B)$.

    This observation tells us that we can permute columns in $A$ the same way as we permute
    rows in $B$ and the result will be the same. This is the key to the balanced SUMMA3D algorithm.
    I told this to my friend in a tram and he decided to give it a try -- look for
    Mieszko Grodzicki's implementation.

    The problem that we are now facing is to choose the permutation matrix $P$.
    We can try to develop state of the art algorithm to find the best permutation matrix,
    but we can also use the following heuristic: chose $P$ randomly.
    Random distribution should be more or less balanced,
    but next problem arises -- data can be cluttered in the first rows of $A$.
    In order to balance this, we would have to choose two permutation matrices $Q_A, Q_B$,
    and permute rows in $A$ and columns in $B$.

    The final formula is
    $$
    C = Q_A^T \cdot (Q_A \cdot A \cdot P^T) \cdot (P \cdot B \cdot Q_B) \cdot Q_B^T.
    $$

    \section{ Testing }

    The code is tested using Google Test framework for unit tests,
    but also is tested on random sparse matrices generated using some
    Python scripts. Both algorithm were tested on hundreds of matrices
    and on hundreds of different process grids.

    Local testing bench allowed me to test at most on 10 processes,
    but some MPI magic unlocked oversubscription on the local machine,
    and allowed to test on over a hundred processes.

    This approach is better than spamming the cluster with small jobs.

    Later, when the code was ready, I tested it on the cluster using
    the provided scripts. Both performance and correctness tests were
    performed.

    \section{ Performance }

    The performance of my implementation is not the best, but
    at least seems correct. I really wanted to implement the balanced
    version, but the end of the semester was a bit overwhelming.

    The performance can be seem on the following plots. There are two types
    of time measurements: total time and algorithm time.
    The total time also includes the time of reading the matrices from the disk,
    which (same as with the CUDA task) is a small bottleneck.

    \plottimes
        {algorithm-time-ms_speedup_medium}
        {algorithm-time-ms_speedup_large}
        {total-time-ms_speedup_medium}
        {total-time-ms_speedup_large}

\end{document}
